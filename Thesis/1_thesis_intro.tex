% !TEX root = Thesis.tex

%==============================================================================
\chapter{Introduction}
\label{chap:intro}
%==============================================================================

The first theorization of an ultracold ensemble of atoms was performed by Satyendra Nath Bose and Albert Einstein in 1924 during a series of publications where the concept of what is today known as a \acf{bec} was developed \cite{Bose1924, Einstein1924, Einstein1925}. At the time, this was a new state of matter formed by bosonic particles occupying the energetic ground state macroscopically for close to absolute zero temperatures. It took more than 70 years to obtain this theorised state of matter in an experiment. The first \acp{bec} were generated in 1995 by several research groups for three different chemical elements: rubidium \cite{Davis1995}, sodium \cite{Anderson1995} and lithium \cite{Bradley1995}. The following years lead to a rising interest of this exotic state of matter due to their quantum properties that allow to describe the system of particles by using just the coherent wave function of a single-particle. Due to this, \acp{bec} for many other elements were achieved. Some examples are: alkali metals like strontium \cite{Stellmer2009}, and Lanthanides like ytterbium \cite{Takasu2003}, dysprosium \cite{Lu2011} and erbium \cite{Aikawa2012}. Even in non-atomic bosonic particles like photons \cite{Klaers2010}.

Nowadays, there has been increasing interest in the condensation of atoms belonging to the lanthanide group. This is due to two main reasons: the first one being the large magnetic moment that these elements normally have, which increases the effect of dipole-dipole interaction \cite{Aikawa2012,Baier2018}. The second reason will be more relevant for this experiment and is based on the fact that these atoms normally have a non-vanishing electronic angular momentum in their energetic ground state. For the case of erbium, the orbital angular momentum has a value of $L=5$ for the ground state. This allows to use Raman transitions between Zeeman sublevels of the ground state in the fine structure scheme. Increasing the available detuning ranges, which enables the possibility of using large values for the detuning. This reduces the photon scattering rates and increases the coherence times for the \ac{bec} \cite{Grimm2000}.

Raman manipulation permits the use of a method called phase imprinting, which generates synthetic magnetic fields and have been theorised to be achievable with lanthanide atoms \cite{cui2013synthetic}. If these synthetic fields were generated with enough strength, it would enable the fractional quantum Hall regime for neutral atoms. This could have mayor implications possibly enabling studies of topological quantum computation. The synthetic magnetic fields have already been observed as vortices structures inside the \ac{bec} for rubidium \cite{Lin2009}.

%%% Local Variables: 
%%% mode: latex
%%% TeX-master: "Thesis"
%%% End: 
