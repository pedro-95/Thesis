% !TEX root = Thesis.tex

%==============================================================================
\chapter{Introduction}
\label{chap:intro}
%==============================================================================

The first theorization of an ultracold ensemble of atoms was performed by Satyendra Nath Bose and Albert Einstein in 1924 during a series of publications where the concept of what is today known as a \acf{bec} was developed \cite{Bose1924, Einstein1924, Einstein1925}. At the time, this was a new state of matter formed by bosonic particles occupying the energetic ground state macroscopically for close to absolute zero temperatures. It took more than 70 years to obtain this theorised state of matter in an experiment. The first \acp{bec} were generated in 1995 by several research groups for three different chemical elements: rubidium \cite{Davis1995}, sodium \cite{Anderson1995} and lithium \cite{Bradley1995}. The following years lead to a rising interest of this exotic state of matter due to their quantum properties that allow to describe the system of particles by using just the coherent wave function of a single-particle. Due to this, \acp{bec} for many other elements were achieved. Some examples are: alkali metals like strontium \cite{Stellmer2009}, and Lanthanides like ytterbium \cite{Takasu2003}, dysprosium \cite{Lu2011} and erbium \cite{Aikawa2012}. Even in non-atomic bosonic particles like photons \cite{Klaers2010}.

Nowadays, there has been increasing interest in the condensation of atoms belonging to the lanthanide group. This is due to two main reasons: the first one being the large magnetic moment that these elements normally have, which increases the effect of dipole-dipole interaction \cite{Aikawa2012,Baier2018}. The second reason will be more relevant for this experiment and is based on the fact that these atoms normally have a non-vanishing electronic angular momentum in their energetic ground state. For the case of erbium, the orbital angular momentum has a value of $L=5$ for the ground state. This allows to use Raman transitions between Zeeman sublevels of the ground state in the fine structure scheme. Increasing the available detuning ranges, which enables the possibility of using large values for the detuning. This reduces the photon scattering rates and increases the coherence times for the \ac{bec} \cite{Grimm2000}.

Raman manipulation permits the use of a method called phase imprinting, which generates synthetic magnetic fields and have been theorised to be achievable with lanthanide atoms \cite{cui2013synthetic}. If these synthetic fields were generated with enough strength, it would enable the fractional quantum Hall regime for neutral atoms. This could have mayor implications possibly enabling studies of topological quantum computation. The synthetic magnetic fields have already been observed as vortices structures inside the \ac{bec} for rubidium \cite{Lin2009}. However, the strength of these fields was limited by the coherence time of the available transitions in this element. As said, the use of erbium permits longer coherence times, which could generate larger synthetic magnetic fields and possibly enable the fractional quantum Hall regime.

This thesis aims to study an Raman manipulation set-up formed by two counter-propagating beams detuned with respect to the \SI{841}{\nano\meter} erbium transition. The main part will consists on the characterization of Raman transitions between the momentum space for the energetic ground state of erbium. This corresponds to the atomic \ac{bec} diffraction with the Raman beams forming and optical lattice. This results in the generation of different momentum orders, similar to the well-known process of light diffraction. After this, the main focus will be the achievement of Raman transitions into the spin-momentum configuration between different sublevels, generated by Zeeman splitting of the ground energy level of the erbium \ac{bec}. This achievement represents the ground work for the future realisation of strong enough synthetic fields to reach the fractional quantum Hall regime. However, in order to obtain the Raman manipulation of internal spin states, an additional experiment with \ac{rf} transitions was required. Its main purpose was to characterize and prepare the magnetic fields causing the Zeeman splitting of the ground state.

Therefore, the thesis it is divided in different chapters according to its content. Chapter \ref{chap:erbium_bec} shows the properties of erbium, introduces the basic theory of an atomic \ac{bec} and briefly describes the experimental set-up used to achieve and measure an erbium \ac{bec}. Chapter \ref{chap:one_dimensional_lattices} gives the theoretical basis for the diffraction of a \ac{bec} with an optical lattice. Chapter \ref{chap:raman_manipulation} describes the \ac{rf} transitions experiment together with a basis of the Raman manipulation of the spin-momentum space. Chapter \ref{chap:results_and_discussion} shows the measured and analysed results. And finally, \ref{chap:outlook} serves as a conclusion and outlook to the thesis.

%%% Local Variables: 
%%% mode: latex
%%% TeX-master: "Thesis"
%%% End: 
