% !TEX root = Thesis.tex

%==============================================================================
\chapter{Diffraction of a \acl{bec} with a one-dimensional Optical lattice}
\label{chap:one_dimensional_lattices}
%==============================================================================
In ultracold atoms physics, Optical lattices are defined as periodic standing wave potentials generated by interfering laser beams. Its use allows to replicate results from solid state physics, with the interaction between an ultracold atomic ensemble and an optical lattice being equivalent to the role of electrons in an atomic lattice \cite{Lewenstein2007, Bloch2008}.  However, the use of ultracold systems presents some advantages when comparing with those used in solid state physics. The main one being the capability of changing the optical lattice properties, by simply adjusting the intensity and phase of the laser beams forming it. An additional advantage of ultracold atomic systems is the non-existence of crystal defects, which can be a big source of noise in solid states systems \cite{VanDerZiel1978}. This experiment will seek to study how an erbium \ac{bec} behaves when interacting with a one-dimensional optical lattice. In this chapter, an introduction describing the theory behind optical lattices will be shown. After this, it follows a brief description of the implemented set up to form the lattice by using the \SI{841}{\nano\meter} erbium transition (see Table \ref{tab:Transitions}).

\section{Theoretical description of an Optical lattice}

As previously mentioned, generating an optical lattice requires the use of two interfering laser beams. Taken individually, each would interact with the atomic cloud like a typical \acf{odt}. However, 

%%% Local Variables: 
%%% mode: latex
%%% TeX-master: "Thesis"
%%% End: 
