% !TEX root = Thesis.tex

%==============================================================================
\chapter{Conclusion and outlook}
\label{chap:outlook}
%==============================================================================

In the present master thesis, several effects related to the energetic manipulation of an ultracold ensemble of erbium atoms were studied. The main part is related to the study of optical manipulation with the use of a lattice potential generated by two counter-propagating light beams. These laser beams were detuned with respect to a narrow line erbium transition near \SI{401}{\nano\meter}, which generates 2-photon Raman transitions into different energetic states. Initially the process was used only for transitions between the momentum space for the same internal state of erbium. Leading to the study of the Bragg and Raman-Nath regimes that allowed to characterize the experimental set-up and the interaction process between erbium and light.

The following part consists in the use of Raman transitions along the spin-momentum space, which is generated by Zeeman splitting taking place in the ground state of erbium. In order to achieve it, the magnetic fields of the experiment have been characterized and adjusted with the use of \acl{rf} transitions and a Stern-Gerlach force. This way the 2-photon detuning condition between Raman beams can be estimated, leading to the achievement of 2-photon transitions between Zeeman and momentum states.

The implemented experimental set-up is the ground work for the generation of synthetic magnetic fields in ultracold erbium clouds. A process that could unravel some limitations of neutral atomic systems, by generating an artificial Lorentzian force like in charged particles. Possibly leading to a quantum Hall regime that could enable studies of topological quantum computation \cite{Lin2009}.



%%% Local Variables: 
%%% mode: latex
%%% TeX-master: "Thesis"
%%% End: 
